\documentclass{article}

\usepackage[utf8]{inputenc}

\title{UE 3I013 : Projet Recherche\\
    SuperViseur\\
    Introduction au sujet}

\author{Basile Pesin}

\begin{document}

\maketitle
\newpage

\section{Sujet de recherche}
\subsection{Motivation} 
L'activité des enseignants durant leur leçons est demeuré pendant longtemps un sujet peu exploré, en particulier a cause du manque de technologies ne pertmettant pas d'obtenir des données en quantité suffisantes. En effet suivre le comportement de l'enseignant et des élèves en temps réel et avec une grande précision demanderait un grand nombre d'observateurs, durant des durées importantes.\\
La technologie de l'Eye Tracking a apporté un début de réponse à ce problème. En 2015, une étude \cite{OrchestrationLoad} a montré que les appareils d'Eye Tracking étaient devenus suffisement portatifs pour pouvoir être utilisés sans gène pendant un cours. Cette étude se concentrait sur la "charge cognitive" des enseignants durant différentes activités, et a permis de montrer, entre autre, que la charge était plus intense lors de l'orchestration d'activités au niveau de la classe, ou lors d'une discussion "en tête a tête" avec un élève (probablement pour jauger la compréhension de ce dernier). De même, l'étude a aussi montré que les épisodes de "haute charge cognitive" étaient plus concentrées chez un professeur novice que chez un professeur expérimenté.\\
Ce que cette étude n'a pas étudié cependant, et que la technologie d'Eye Tracking rends accessible, c'est la "quantité d'attention" donnée à l'enseignant à tel ou tel élève. C'est à cette question que le projet SuperViseur \cite{SuperViseur} tente de répondre.

\subsection{Objectifs}

\section{Taches a réaliser}

\bibliography{bibliographie}{}
\bibliographystyle{plain}
\end{document}
