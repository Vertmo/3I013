\documentclass{article}

\usepackage[utf8]{inputenc}
\usepackage{hyperref}
\usepackage{graphicx}
\usepackage[french]{babel}

\graphicspath{{ressources/}}

\title{UE 3I013 : Projet Recherche\\
    Encadrante : Vanda Luengo\\
    SuperViseur : Tests Utilisateurs}

\author{Basile Pesin}

\begin{document}

\maketitle
\newpage

\section{Cible utilisateur}
Les utilisateurs du projet SuperViseur sont de deux catégories : la communauté scientifique, et les enseignants / futurs enseignants en formation. En ce qui concerne la communauté scientifique, qui est habitué à manipuler des données et des visualisations diverses, les problèmes d'ergonomies ne devraient pas être trop importants. En revanche, il faudra particulièrement tester l'ergonomie du site pour les enseignants (en particulier ceux en formation), y compris ceux qui pourraient ne pas avoir une grand expérience des outils informatique.\\
On voudra donc réaliser le test sur un échantillon d'enseignants débutants ou en formation, ainsi que quelques membres de la communauté scientifique. On pourra éventuellement s'adresser aux étudiants de la mineure EDS (Enseignement et Didactique des Sciences) de Sorbonne Université.

\section{Objectifs utilisateurs}
On pourra présenter une liste de questions (portant sur des observations à faire sur le site) aux utilisateurs. On mesurera l'exactitude des réponses, ainsi que le temps nécessaires pour les trouver. On observera aussi le comportement et les réactions des utilisateurs durant les tests.\\
Ci dessous une liste des questions qu'on pourra poser :
\begin{itemize}
    \item Quel est le premier élève que regarde l'enseignant 4 durant le cours ?
    \item Quel est l'enseignant qui distribue le plus son attention (coefficient GINI le moins élevé) ?
    \item Combien y'a t-il d'élèves ayant des besoins particuliers dans la classe de l'enseignant 1 ?
    \item Quel est l'élève le plus regardé par l'enseignant 3 ? 
\end{itemize}

\end{document}

