\documentclass{article}

\usepackage[utf8]{inputenc}
\usepackage{hyperref}
\usepackage{graphicx}
\usepackage[french]{babel}

\graphicspath{{ressources/}}

\title{UE 3I013 : Projet Recherche\\
    Encadrante : Vanda Luengo\\
    SuperViseur : Rapport}

\author{Basile Pesin}

\begin{document}

\maketitle
\newpage

\section{Rappel des problématiques principales, hypothèses}


\section{Choix techniques}
Afin de pouvoir aisément présenter les résultats de l'étude à la communauté scientifique, ainsi qu'a la communauté éducative à des fins pédagogiques, il a été décidé de rendre les visualisations disponibles sous forme d'un site web. Même si cela s'éloigne légèrement du fonctionnement d'\href{https://undertracks.imag.fr}{Undertracks}, ou les opérateurs sont écrits en Python, on peu imaginer mettre à l'avenir en place un système similaire à celui \href{http://superviseur.lip6.fr}{superviseur.lip6.fr}, ou des opérateurs Python générent des morceaux de codes HTML/Javascript à "imbriquer" dans le reste du code.

\subsection{Bibliothèques}
Afin de simplifier le développement, on a utilisé plusieurs bibliothèques de fonctionnalités Javascript :
\begin{itemize}
    \item \href{https://jquery.com/}{jQuery} qui permet de facilement manipuler le DOM (Document Object Model) HTML. Cette bibliothèque permet également de simplifier les requètes ajax (qui permettent entre autre de charger les données de l'étude). De plus, c'est une dépendance de la bibliothèque suivante.
    \item \href{https://semantic-ui.com/}{Semantic UI} est un framework HTML/CSS, permettant donc de créer un site web d'aspect convenable rapidement.
    \item \href{http://svgjs.com/}{SVG.js} a été choisi pour construire les représentations spatiales et circulaires. Comme son nom l'indique, cette bibliothèque manipule des SVG (Scalable Vector Graphics). (au départ, et comme indiqué dans le document de compréhjsion, on souhaitait utiliser la biblioithèque \href{https://p5js.org/}{p5.js}, mais celle-ci était beaucoup moins pratique pour gérer le survol de souris, et qui plus est elle tendait à créer des fuites de mémoires).
    \item \href{http://www.chartjs.org/}{Chart.js} est une bibliothèque permettant d'afficher des graphiques interactifs. On l'utilise principalement pour résumer les résultats sur la page d'accueil du site.
\end{itemize}

\subsection{Données}
\subsubsection{Tables}
Comme expliqué dans le document de compréhension, les données à traiter pour ce projet se présentent sous la forme de trois fichier csv exportés depuis la plateforme \href{https://undertracks.imag.fr/}{UnderTracks} : \textit{users.csv} qui contient la liste des élèves, \textit{events.csv} qui contient la timeline des évènements, et \textit{context.csv}, qui contient la description des constantes et acronymes utilisés.\\
Ces 3 fichiers sont placés dans le dossier \textit{/data} du projet. On aurait préféré pouvoir charger les données directement depuis la plateforme UnderTracks (via une REST API par exemple) afin de ne pas avoir à stocker les fichiers dans le dépot, mais cela n'est malheureusement pas encore possible.\\
Dans le cadre de ce projet, on utilisera principalement la table des utilisateurs et celle des évènements. On charge ces deux tables en même temps que le site au moyen de requètes Ajax, et on garde les informations chargées en mémoire. Dans le cas des utilisateurs ayant une connexion lente, le chargement de ces tables (en particulier la tables des évènements, qui compte plus de 25000 lignes, ou 6.7Mo de données) peut présenter un léger ralentissement. Heureusement, le fait qu'on ne charge ces données qu'une fois par visite de la page ainsi que le cache du navigateur permettent de réduire le nombre de chargements et donc ce problème.

\subsubsection{Anonymisation des données}
Le projet SuperViseur impliquant la récolte de données sur des élèves de primaire (CP, CE1, CE2) il est très important de correctement anonymiser les données avant de les rendre publiques. Cela signifie, entre autres, que les vidéos enregistrés par le système d'eye tracking ne sont pas accessibles au public, et que les prénoms des élèves ont évidemment été supprimés de la table des utilisateurs avant même la mise en ligne sur UnderTracks. Le problème est que ces prénoms se trouvent aussi dans la verbalisation de l'enseignant (qui s'adresse à ses élèves par leurs prénoms). Même si cela ne permet pas directement d'identifier quel élève porte quel prénom, on peut, en mettant en relation les élèves regardés et ceux nommés par l'enseignant, parvenir à identifier les élèves.\\
On a donc réalisé un petit script permettant de remplacer les prénoms des élèves par leurs numéros dans la verbalisation de l'enseignant. En plus d'anonymiser les données, cela a l'avantage de permettre de mieux visualiser les relations entre les élèves nommés et les élèves regardés.

\subsubsection{Position de l'enseignant}
Une information utile dans le cadre de nos visualisations spatiales est la position de l'enseignant dans la classe. En effet, on peut faire l'hypothèse que le regard de l'enseignant se pose plus souvent sur les élèves proches de lui. Le problème est que cette information n'est pas présente dans la transcription des données actuellement disponible (mais elle le sera à l'avenir).\\
Afin de pouvoir malgré tout tester cette fonctionnalité de nos visualisations, on a donc crée une version "factice" de cette donnée, en ajoutant à la table des évènements deux colonnes posX et posY indiquant respetivement la position en X et en Y de l'enseignant (pour les questions de direction des axes, d'origine et d'echelle, on a gardé la convention utilisée pour noter la position des élèves). Dans nos données factices, l'enseignant suit donc une marche aléatoire en deux dimensions : $(posX, posY) \in [0, 40]^2$.

\section{Réponses}

\section{Conclusion}

\bibliography{bibliographie}{}
\bibliographystyle{plain}
\end{document}

