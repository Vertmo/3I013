\documentclass{article}

\usepackage[utf8]{inputenc}
\usepackage[french]{babel}

\title{UE 3I013 : Projet Recherche\\
    Encadrante : Vanda Luengo\\
    SuperViseur : Protocole de test}

\author{Basile Pesin}

\begin{document}

\maketitle

\section{Objectifs}
On prends en note et évalue:
\begin{itemize}
    \item Réussite: au moins 4 taches sur les 5
    \item Temps : moins de 15min au total (pour un utilisateur sachant se servir correctement de l'outil informatique)
    \item Erreurs: moins de 3 par tache
    \item Incompréhensions: 1 ou moins par tache
    \item Soupirs: moins de 3
\end{itemize}

\section{Présentation du test}
On lit le texte suivant au volontaire:\\
"L'objectif de ce test est d'évaluer l'ergonomie et l'usabilité du site web SuperViseur. Ce site est dedié à la visualisation de données récoltées durant des séances de cours d'enseignants de primaire. Ces données ont été récoltées au moyen de dispositif occulométriques (qui permet de relever la position du regard au cours de la séance), ainsi que grace a des observateurs humains. La verbalisation de l'enseignant, ainsi que les activités pédagogiques, et des informations sur les élèves sont également disponibles.
Ce test à pour but d'évaluer uniquement la qualité de l'interface, et non vos performances. Nous allons vous demander d'accomplir une série de taches en vous servant du site. Nous souhaitons que vous nous fassiez part de votre ressenti, de vos incompréhensions et des raisons de vos actions tout au long de vos manipulations. Encore une fois, n'hésitez pas à dire tout ce qui vous passe par la tête, le but est de trouver les défauts du site."

\section{Explication des sigles}
\begin{itemize}
    \item TDOP : Teaching Dimensions Observation Protocole : classification des différent types d'activités pédagogiques.
    \item GINI : coefficient mesurant l'équité de la répartition d'une ressource. Varie entre 0 et 1. Plus il est bas, plus la répartition est équitable.
\end{itemize}

\section{Taches à réaliser}
\begin{enumerate}
    \item Vous souhaitez trouver les premiers élèves regardés par chacun des enseignants, et leurs niveaux respectifs en Français. Vous remarquez quelque chose (quoi ?) et vous souhaitez comparer cela avec la distribution globale de l'attention en fonction du niveau en Francais.

    \item Identifier les enseignants experts et novices. Lesquels ont l'attention la plus fragmenté ? Lesquels distribuent le plus également leur attention ? (coefficient GINI le moins élevé)

    \item Trouver les élèves ayant des besoins particuliers, et ces besoins. Sont ils plus regardés que les autres ?

    \item Observer le TDOP au début du cours de l'enseignant 1. Observer les TDOPs durant le reste du cours. Est-ce un cours de Français ou de Maths ?

    \item Observer le déplacement de l'enseignant 4. A coté de quel élève se trouve-t-il à 21min?
\end{enumerate}

\section{Debriefing}
On interroge en face à face le volontaire sur:
\begin{itemize}
    \item Son ressenti général (question ouverte, laisser parler)
    \item Les difficultés particulières rencontrées, les incompréhensions
    \item Les fonctionnalités (trop/pas assez)
    \item Autres remarques
\end{itemize}

Et on remercie évidemment le volontaire à la fin du test!

\end{document}

