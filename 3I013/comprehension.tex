\documentclass{article}

\usepackage[utf8]{inputenc}

\title{UE 3I013 : Projet Recherche\\
    SuperViseur\\
    Introduction au sujet}

\author{Basile Pesin}

\begin{document}

\maketitle
\newpage

\section{Sujet de recherche}
\subsection{Etat de l'art}
L'activité des enseignants durant leur leçons est demeuré pendant longtemps un sujet peu exploré, en particulier a cause du manque de technologies ne pertmettant pas d'obtenir des données en quantité suffisantes. En effet suivre le comportement de l'enseignant et des élèves en temps réel et avec une grande précision demanderait un grand nombre d'observateurs, durant des durées importantes.\\
La technologie de l'Eye Tracking a apporté un début de réponse à ce problème. En 2015, une étude \cite{OrchestrationLoad} a montré que les appareils d'Eye Tracking étaient devenus suffisement portatifs pour pouvoir être utilisés sans gène pendant un cours. Cette étude se concentrait sur la "charge cognitive" des enseignants durant différentes activités, et a permis de montrer, entre autre, que la charge était plus intense lors de l'orchestration d'activités au niveau de la classe, ou lors d'une discussion "en tête a tête" avec un élève (probablement pour jauger la compréhension de ce dernier). De même, l'étude a aussi montré que les épisodes de "haute charge cognitive" étaient plus concentrées chez un professeur novice que chez un professeur expérimenté.\\
Ce que cette étude n'a pas étudié cependant, et que la technologie d'Eye Tracking rends accessible, c'est la "quantité d'attention" donnée à l'enseignant à tel ou tel élève. En 2014, une étude de Van Den Bogert \cite{VanDenBogert} avait déjà tenté de répondre à cette question. Cependant, le protocole expérimental choisi était de faire regarder a des enseignants (20 novices et 20 experts) des vidéos de cours pré-enregistrées. Malgré le manque de réalisme de cette méthode, l'étude a quand même réussi a confirmer que les novices fixaient leurs regards sur plus d'éléments différents (leur regard "papillone" plus), et que les experts étaient capables d'observer plus d'élèves.\\
L'année suivante, Cortina et son équipe \cite{Cortina} tentent la même expérience, mais dans un contexte réel cette fois. Ils montrent une fois de plus que l'attention des enseignants experts est plus équitablement distribuée entre les élèves que celle des novices. De plus, en utilisant le CLASS (Classroom Assessment Scoring System), un outil de la "qualité" de l'environnement d'étude, ils montrent aussi que cette distribution de l'attention est corrélée positivement au score CLASS.\\
Ce à quoi ces études ne s'intéressent pas cependant, c'est l'identification des groupes d'élèves recevant plus d'attention. C'est à cette question que s'intéresse, en 2016, le projet SuperViseur \cite{SuperViseur}.

\subsection{Questionnements et Hypothèses}
Le but de l'étude menée est double : premièrement, déterminer la "quantité d'attention" attribuée aux élèves, en fonction du niveau percu de l'élève, et tenter de déterminer des groupes sur lesquels l'attention se focaliserait particulièrement. De plus, en réutilisant le CLASS vu plus haut, et au moyen d'observateurs dans la classe, l'étude analyse aussi les relations entre l'attention donnée par l'enseignant, et les interactions avec ses élèves en resultant (rétroactions).\\
A partir de ces questions, voici les hypothèses formulées par l'étude :
\begin{itemize}
  \item Comme dans l'etude de Cortina \cite{Cortina} les enseignants qui obtiendront les résultats les plus élevés au CLASS seront ceux distribuant le plus équitablement leur attention. Il est important d'ajouter que cette distribution d'attention doit étre suivie de rétroactions des enseignant : en effet ce sont ces rétroactions qui font varier l'état de la classe, et donc le CLASS. Cette hypothèse peut donc devenir : une attention distribuée plus équitablement entraine des rétroactions de meilleure qualités, ce qui a pour conséquence directe de meilleurs résultats au CLASS.
  \item Il existe un groupe sur lequel l'enseignant va se concentrer tout particulièrement. Ce groupe sera composé des élèves percus par l'enseignant comme ayant des difficultés comportementales.
  \item Il existe une différence entre les comportements des enseignants novices et ceux des experts. En particulier, comme vu précédemment, les novices distribueront leur attention moins équitablement, et donc d'après la première hypothèse, obtiendront des résultats CLASS plus faible.
\end{itemize}

\subsection{Méthodologie et données récoltées}
Comme vu plus haut, les données récoltées "pendant l'action" sont de deux natures:

\section{Travail à réaliser}

\bibliography{bibliographie}{}
\bibliographystyle{plain}
\end{document}
